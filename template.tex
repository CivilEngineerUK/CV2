%%%%%%%%%%%%%%%%%%%%%%%%%%%%%%%%%%%%%%%%%
% Twenty Seconds Resume/CV
% LaTeX Template
% Version 1.0 (14/7/16)
%
% This template has been downloaded from:
% http://www.LaTeXTemplates.com
%
% Original author:
% Carmine Spagnuolo (cspagnuolo@unisa.it) with major modifications by 
% Vel (vel@LaTeXTemplates.com)
%
% License:
% The MIT License (see included LICENSE file)
%
%%%%%%%%%%%%%%%%%%%%%%%%%%%%%%%%%%%%%%%%%

%----------------------------------------------------------------------------------------
%	PACKAGES AND OTHER DOCUMENT CONFIGURATIONS
%----------------------------------------------------------------------------------------
\documentclass[letterpaper]{twentysecondcv} % a4paper for A4
\usepackage{tabularx}
\usepackage{graphicx, array, blindtext, hyperref, calc, multirow, enumitem}
\hypersetup{
    colorlinks=true,
    linkcolor=black,
    filecolor=black,      
    urlcolor=black,
}

\newlength\myheight
\newlength\mydepth
\settototalheight\myheight{Xygp}
\settodepth\mydepth{Xygp}
\setlength\fboxsep{0pt}
\newcommand*\inlinegraphics[1]{%
  \settototalheight\myheight{Xygp}%
  \settodepth\mydepth{Xygp}%
  \raisebox{-\mydepth}{\includegraphics[height=1.2\myheight]{#1}}%
}
 
\urlstyle{same}

%----------------------------------------------------------------------------------------
%	 PERSONAL INFORMATION
%----------------------------------------------------------------------------------------

% If you don't need one or more of the below, just remove the content leaving the command, e.g. \cvnumberphone{}



\cvname{Michael Rustell EngD CEng MICE} % Your name
\cvjobtitle{Senior Maritime Engineer}
%\cvjobtitle{\begin{itemize}
%\item Structural Dynamics 
%\item Machine Learning
%\item Statistics \& Data Science 
%\end{itemize}} % Job title/career

\cvlinkedin{https://linkedin.com/in/michaelrustell}
\cvnumberphone{+44 7591454736} % Phone number
\cvsite{http://mysite\textasciitilde mjr} % Personal website
\cvmail{michaelrustellresearch} % Email address

%----------------------------------------------------------------------------------------

\begin{document}

\makeprofile % Print the sidebar

\section{Summary}

Technically adept and competent Chartered Civil Engineer with skills particularly in dynamics, finite element programming and developing statistical models. 
Extensive maritime, data analysis and machine learning experience within an industrial environment. Recent projects have included seismic analysis lead on nuclear projects including soil-structure interaction and controlled ductile response and vessel impact specialist for the Thames Tideway Tunnel. 

%----------------------------------------------------------------------------------------
%	 EDUCATION
%----------------------------------------------------------------------------------------
\section{Education}

\renewcommand{\arraystretch}{0.5}
\begin{table}[ht]
%\setlength{\extrarowheight}{.5em}
\centering
\begin{tabular}{p{0.12\textwidth} p{0.68\textwidth} m{0.2\textwidth}}
% surrey EngD
2015 & \textbf{Doctor of Engineering} in \href{http://epubs.surrey.ac.uk/812133/1/Rustell.\%20M.\%20Thesis\%202016.pdf}{\textbf{\textit{Stochastic Optimisation of LNG Terminal Infrastructure}}} 

Using \textit{machine learning} and \textit{statistics} to simultaneously minimise cost, risk and berth downtime.

& \href{http://www.surrey.ac.uk}{\includegraphics[width = 0.9\linewidth]{Figures/surreylogo.png}}\\

%ABDN
2015 & \textbf{PGCert} in \textbf{\textit{Oil and Gas Structural Engineering}} 

Dynamics, jackets, topsides, steel connections etc.

& \href{https://www.abdn.ac.uk/study/postgraduate-taught/degree-programmes/217/oil-and-gas-structural-engineering/}{\includegraphics[width = 0.9\linewidth]{Figures/aberdeen-university-logo.jpg}}\\

%Plym
2010/ 2009 & \textbf{MSc \& BSc} in \textbf{\textit{Civil Engineering}} 
\begin{itemize}\setlength\parskip{0pt} 
\item  Scott Wilson Award for \textit{Best final year project}
\item IStructE award for \textit{Best Presentation} 
\item IET Award for \textit{Best Engineering Project} 
\end{itemize} &
\href{https://www.plymouth.ac.uk/courses/postgraduate/msc-civil-engineering}{\includegraphics[width = 0.9\linewidth]{Figures/plym2.png}}\\

\end{tabular}
\end{table}
%----------------------------------------------------------------------------------------
%	 EXPERIENCE
%----------------------------------------------------------------------------------------

\section{Experience}

\begin{table}[ht]
\centering
\begin{tabular}{p{0.12\textwidth} p{0.68\textwidth} m{0.2\textwidth}}

% AECOM
2015 - 

Current & \textbf{Senior Maritime Engineer}, \textit{Ports \& Marine}
\begin{itemize}[topsep=0pt]
\item Stochastic vessel impact \& seismic analysis
\item Development of R, MATLAB, C++ and Python tools for statistical analysis and optimisation
\end{itemize}
& \href{http://www.aecom.com/markets/transportation/ports-marine/}{\includegraphics[width = 0.75\linewidth]{Figures/AECOM.jpg}}
\\

% HRW
2011-2015 & \textbf{Research Engineer}, \textit{Engineering Group} 
\begin{itemize}[topsep=0pt]
\item PIANC \textit{De Paepe-Willems Award} 2014  
\item COPRI \textit{Best Student Paper} Award 2013
\item RAENG Panasonic Trust awardee
\item Two-time recipient of UGPN Travel Award
\end{itemize}

& \href{http://www.hrwallingford.com/expertise/port-harbour-development}{\includegraphics[width = 0.75\linewidth]{Figures/HR-Wallingford-logo.jpg}}\\

\end{tabular}
\end{table}

\section{Key Projects}

\textbf{Thames Tideway Tunnel, UK} – Developed statistical model to estimate vessel impact for all critical infrastructure in central 10km section. Continued involvement as expert in vessel impact.

\textbf{Nuclear (Confidential), UK} – Developed analysis methodology for seismic loading including soil-structure interaction and controlled ductile response. Coordinated a team of 9 engineers. Lead author on key reports that undergo review by government client. Design package lead for one of three key structures.

\textbf{Port (Confidential), UK} – Designers representative on a live container port undergoing significant repairs on a defective quay wall. Planned and oversaw testing and research of materials and products in unconventional applications. Dealt with design issues related to hydrogen embrittlement.


\section{Awards}



\end{document} 
